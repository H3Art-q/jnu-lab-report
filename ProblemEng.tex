% 关闭 ctex 的中文特性
\PassOptionsToPackage{scheme=plain}{ctex}
\documentclass{ExpReport}

\course{数值计算实验} %课程名称
\director{Your Director} %指导老师
\author{学生A,学生B} %学生姓名
\classroom{N117} %实验地点
\title{Computing Problems} %实验名称
\expno{080812345601} %实验编号
\exptype{} %实验项目类型,没有要求的话请留空
\authorno{2020101234,2019051234} %学号
\college{信息科学技术学院} %学院
\department{计算机科学} %系
\major{计算机科学与技术} %专业

\date{2021年3月1日 $\sim$ 2021年3月1日} %时间,可删掉,删掉后可以自动生成时间

\temp{} %温度,可留空
\humid{} %湿度,可留空

% 设置章节编号为罗马数字,符合英文习惯
\renewcommand\thesection{\Roman{section}.}

\begin{document}
    % 因为 ctex 的中文特性被关闭,行间距由 1.3 变回 1.0 ,所以这里我们要重新调整标题区的表格间距
    \begingroup
    \renewcommand\arraystretch{1.65}
    \maketitle
    \endgroup
    % 请不要移除上述代码

    \section{Problem}

    Let A be the 1000 $\times$ 1000 matrix with entries $A(i, i) = i, A(i, i + 1) =
A(i + 1, i) = \frac{1}{2}, A(i, i + 2) = A(i + 2, i) = \frac{1}{2}$ for all i that fit within the matrix.
    \begin{itemize}
        \item Solve the system with $Ax = [1, 1, · · · , 1]^T$ by the following methods in 15 steps:
        \begin{enumerate}
            \item The Jacobi Method;
            \item The Gauss-Seidel Method;
            \item SOR with $\omega = 1.1$;
            \item The Conjugate Gradient Method;
            \item The Conjugate Gradient Method with Jacobi preconditioner.
        \end{enumerate}
        \item Report the errors of every step for each method.
    \end{itemize}

    \section{Algorithm Summary}

    \subsection{The Jacobi Method}
    The Jacobi Method is a form of fixed-point iteration for a system of equations.

    \begin{itemize}
        \item Let D denote the main diagonal of $A$, $L$ denote the lower triangle of $A$ (entries below the main diagonal), and $U$ denote the upper triangle (entries above the main diagonal).
        \item Then $A = L + D + U$ and the equation to be solved is $L x + Dx + U x = b$. Note that this use of $L$ and $U$ differs from the use in the $LU$ factorization, since all diagonal entries of this $L$ and $U$ are zero.
    \end{itemize}

    
    \section{Experimental Summary}

    In this experiment, we use five methods to solve the system with $Ax = [1, 1, 
    ... , 1]^T$ in 15 steps. We find that the Gauss-Seidel method converges the fastest and the error of the result is the smallest. Next is the Conjugate Gradient method with Jacobi preconditioner, then is the SOR with $\omega=1.1$, next is the Jacobi method. The worst is the Conjugate Gradient method which is far inferior to other methods.

    We notice that matrix A is not A strictly diagonally dominant matrix, but A weakly diagonally dominant matrix. The applicable condition of the first three methods (Jacobi, GAuss-Seidel, SOR) is strictly diagonally dominant matrix, so the first three methods are not suitable for this experiment.
\newpage
\appendix
\section{Appendix: Source Code}
\subsection{Problem.py}
\begin{lstlisting}[language=Python]
import numpy as np

print("Hello")
    
def incmatrix(genl1,genl2):
    m = len(genl1)
    n = len(genl2)
    M = None #to become the incidence matrix
    VT = np.zeros((n*m,1), int)  #dummy variable
    
...
    return M
\end{lstlisting}

\subsection{Problem.m}
\begin{lstlisting}[language=matlab]
for n = 1:2
    for m = 1:3
        fprintf('n = %3u m = %3u \r', n, m)
        % This is a comment
    end
end
\end{lstlisting}

\subsection{Problem.c}
\begin{lstlisting}[language=c]
#include <stdio.h>
int main(){
    printf("Hello world");
    // This is a comment.
}
\end{lstlisting}

\end{document}