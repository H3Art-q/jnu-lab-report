\documentclass{ExpReport}

\course{模拟电子技术实验} %课程名称
\director{赵钱孙,李周吴} %指导老师
\author{学生A,学生B} %学生姓名
\classroom{南海楼} %实验地点
\title{晶体管共射极单管放大器} %实验名称
\expno{080812345601} %实验编号
\exptype{} %实验项目类型,没有要求的话请留空
\authorno{2020101234,2019051234} %学号
\college{信息科学技术学院} %学院
\department{电子} %系
\major{电子科学与技术} %专业

\date{2021年3月1日 $\sim$ 2021年3月1日} %时间,可删掉

\temp{} %温度,可留空
\humid{} %湿度,可留空

\begin{document}
    \maketitle
    
    \section{实验目的}

    \begin{enumerate}
        \item 学会放大器静态工作点的调试方法,分析静态工作点对放大器性能的影
        响。
        \item 掌握放大器电压放大倍数、输入电阻、输出电阻及最大不失真输出电压
        的测试方法。
        \item 熟悉常用电子仪器及模拟电路实验设备的使用。
    \end{enumerate}

    \section{实验原理}

    \subsection{实验原理}
    它的静态工作点可用下式估算:
    $$
    U_B = \frac{R_{B1}}{R_{B1}+R_{B2}}U_{CC}
    $$

    \section{调试分析与测试结果}

    放大器的幅频特性是指放大器的电压放大倍数 $A_U$与输入信号频率 $f$ 之间的关系曲线。$A_{um}$为中频电压放大倍数,通常规定电压放大倍数随频率变化下降到中频放大倍数的 $1/\sqrt{2}$ 倍。

    \begin{table}[htbp]
        \begin{tabular}{ |p{3cm}||p{3cm}|p{3cm}|p{3cm}|  }
            \hline
            \multicolumn{4}{|c|}{Country List} \\
            \hline
            Area Name& ISO ALPHA 2 &ISO ALPHA 3&ISO numeric\\
            \hline
            Afghanistan   & AF    &AFG&   004\\
            Aland Islands&   AX  & ALA   &248\\
            Albania &AL & ALB&  008\\
            Algeria    &DZ & DZA&  012\\
            \hline
       \end{tabular}
    \end{table}
\newpage
\appendix
\section*{附录 (程序清单) }
\subsection{Problem.py}
\begin{lstlisting}[language=Python]
import numpy as np

print("Hello")
    
def incmatrix(genl1,genl2):
    m = len(genl1)
    n = len(genl2)
    M = None #to become the incidence matrix
    VT = np.zeros((n*m,1), int)  #dummy variable
    
...
    return M
\end{lstlisting}

\subsection{Problem.m}
\begin{lstlisting}[language=matlab]
for n = 1:2
    for m = 1:3
        fprintf('n = %3u m = %3u \r', n, m)
        % This is a comment
    end
end
\end{lstlisting}

\subsection{Problem.c}
\begin{lstlisting}[language=c]
#include <stdio.h>
int main(){
    printf("Hello world");
    // This is a comment.
}
\end{lstlisting}

\end{document}